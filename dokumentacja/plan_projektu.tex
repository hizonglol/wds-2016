\documentclass[a4paper]{article}
\usepackage{polski}
\usepackage[utf8]{inputenc}
\usepackage{enumerate}
\usepackage{hyperref}

\title{Wizualizacja danych sensorycznych - projekt}
\date{}

\begin{document}
\maketitle

\begin{enumerate}

\item Temat projektu:

Wizualizacja pogody w Japonii.

\item Wykonawca:

Filip Malinowski 209193

\item Opis projektu:

Projektowany program będzie pobierać pogodę z różnych serwisów pogodowych w Japonii. Dane będzie aktualizował dynamicznie co okres czasu wybrany przez użytkownika, np. co 30 minut, 1 godzinę, 2 godziny. Informacje o pogodzie będą wyświetlane na interaktywnej mapie Japonii. Interaktywność mapy będzie polegać na tym, że domyślnie dla każdej stolicy prefektury będzie wyświetlana temperatura i ikona zachmurzenia. Po kliknięciu na nazwę stolicy prefektury rozwinie się szczegółowa pogoda zawierająca: temperaturę, szansę opadów, ciśnienie, wilgotność,  zachmurzenie, prędkość i kierunek wiatru oraz jeśli będzie dostateczna ilość informacji w serwisach internetowych to poziom promieniowania słonecznego. Mapę będzie można również przesuwać oraz powiększać. Wtedy po przekroczeniu progu powiększenia na mapie pojawią się dodatkowe mniejsze miasta prefektury. W programie będzie również druga zakładka, w której użytkownik będzie mógł wyszukać miasto  i wyświetlić dla niego bardziej szczegółowe informacje na temat pogody niż by otrzymał z mapy. Wyświetlona zostanie prognoza godzinowa lub tygodniowa wszystkich czynników podanych wcześniej do wyświetlania na mapie.

\item Harmonogram:

Harmonogram projektu będzie aktualizowany wraz z każdym osiągnięciem danego terminu oraz z każdorazowym wysłaniem sprawozdania.

\begin{description}
\item[9 kwietnia] – przeprowadzenie badań na temat najlepszych źródeł informacji oraz odpowiedniej struktury programu;

\item[22 kwietnia] – napisanie wersji alpha programu realizującej podstawowe funkcje programu. Podstawowe funkcje programu zostaną określone po osiągnięciu terminu 9 kwietnia;

\item[23 kwietnia] – podsumowanie wstępnej wersji programu oraz ewentualne poprawki w planach projektu.
\end{description}

\item Zarządzanie projektem:
\begin{itemize}
\item Git - do przechowywania i rozwijania dokumentacji oraz oprogramowania związanego z projektem.
\href{https://github.com/hizonglol/wds-2016}{Adres internetowy repozytorium.}
\item LaTeX - do tworzenia dokumentacji projektu
\end{itemize}
\end{enumerate}

\end{document}